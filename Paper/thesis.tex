%%%%%%%%%%%%%%%%%%%%%%%%%%%%%%%%%%%%%%%%%%%%%%%%%%%%%%%%%%%%%%%%%%%%%%%%
%                                                                      %
% LaTeX, FIIW thesis template                                          %
% 28/11/2014 v1.2                                                      %
%                                                                      %
%%%%%%%%%%%%%%%%%%%%%%%%%%%%%%%%%%%%%%%%%%%%%%%%%%%%%%%%%%%%%%%%%%%%%%%%
\documentclass[11pt,a4paper]{report}
% Indien je je thesis recto-verso wil afdrukken gebruik je onderstaande opties i.p.v. bovenstaande
%\documentclass[11pt,a4paper,twoside,openright]{report}

\usepackage[a4paper,left=3.5cm, right=2.5cm, top=3.5cm, bottom=3.5cm]{geometry}
\usepackage[english]{babel}
\usepackage{titlesec}

\usepackage{graphicx}
%\usepackage[latin1]{inputenc}           % om niet ascii karakters rechtstreeks te kunnen inputten
\usepackage[utf8]{inputenc}            % commentarieer deze regel uit als je utf8 encoded files gebruikt in plaats van latin1
\usepackage[authoryear, round]{natbib}
\usepackage{listings}             		% voor het weergeven van broncode
\usepackage{verbatim}					% weergeven van code, commando's, ...
\usepackage[hidelinks]{hyperref}		% maak PDF van de thesis navigeerbaar without boxes
%\usepackage{hyperref}					% maak PDF van de thesis navigeerbaar
\usepackage{url}						% URL's invoegen in tekst met behulp van \url{http://}
\usepackage[small,bf,hang]{caption}     % om de captions wat te verbeteren
\usepackage[final]{pdfpages}            % gebruikt voor het invoegen van het artikel in pdf-formaat
\usepackage{pslatex}					% andere lettertype's dan de standaard types
\usepackage{lipsum}
\usepackage{sectsty}					% aanpassen van de fonts van sections en chapters
%\usepackage[nottoc,numbib]{tocbibind}	% Bibliography mee in de ToC
\usepackage{tikz}
\usepackage[table,xcdraw]{xcolor}
\usepackage{standalone}
\usepackage{enumitem}
\usepackage{multirow}
\usepackage{chngpage}
\usepackage[svgpath=fig/]{svg}
% \usepackage{pdflscape}
\usepackage{rotating}
\usepackage{newfloat}
\usepackage{subcaption}
\DeclareFloatingEnvironment[fileext=frm,placement={!ht},name=Listing,within=chapter]{listing}

\setlist{nosep}
%\usetikzlibrary{external}
\usepackage[acronym]{glossaries}
\usepackage{algorithm}
\usepackage{algpseudocode}
\usepackage{amssymb}% http://ctan.org/pkg/amssymb
\usepackage{pifont}% http://ctan.org/pkg/pifont
\newcommand{\cmark}{\ding{51}}%
\newcommand{\xmark}{\ding{55}}%

%\chapter*{List of abbreviations}

\newacronym{rdf}{RDF}{Resource Description Framework}
\newacronym{r2rml}{R2RML}{Relational Database to RDF Mapping Language}
\newacronym{rml}{RML}{RDF Mapping Language}
\newacronym{shacl}{SHACL}{Shapes Constraint Language}
\newacronym{shex}{ShEx}{Shape Expressions language}
\newacronym{owl}{OWL}{Web Ontology Language}
\newacronym{rdfs}{RDFS}{RDF Schema}
\newacronym{json}{JSON}{JavaScript Object Notation}
\newacronym{xml}{XML}{Extensible Markup Language}
\newacronym{csv}{CSV}{Comma-separated values}
\newacronym{iri}{IRI}{Internationalized Resource Identifier}
\newacronym{uri}{URI}{Uniform Resource Identifier}
\newacronym{url}{URL}{Uniform Resource Locator}
\newacronym{http}{HTTP}{HyperText Transfer Protocol}
\newacronym{html}{HTML}{HyperText Markup Language}
\newacronym{sparql}{SPARQL}{SPARQL Protocol And RDF Query Language}
\newacronym{w3c}{W3C}{World Wide Web Consortium}
\newacronym{pom}{POM}{predicate object map}
\newacronym{om}{om}{Object map}
\newacronym{ai}{AI}{Artificial intelligence}
\newacronym{xsparql}{XSPARQL}{a query language combining XQuery and SPARQL}
\newacronym{d2rml}{D2RML}{Data to RDF Mapping Language}
\newacronym{dm}{DM}{Direct Mapping}
\newacronym{yaml}{YAML}{YAML Ain't Markup Language}
\newacronym{foaf}{FOAF}{Friend of a Friend}
\newacronym{xquery}{XQuery}{XML Query Language}
\newacronym{xslt}{XSLT}{Extensible Stylesheet Language Transformations}
\newacronym{lubm}{LUBM}{Lehigh University Benchmark}
\newacronym{lubm4obda}{LUBM4OBDA}{LUBM for Ontology Based Data Access}
\newacronym{odba}{OBDA}{Ontology Based Data Access}
\newacronym{gtfs}{GTFS}{General Transit Feed Specification}


\allsectionsfont{\sffamily}
\chapterfont{\raggedleft\sffamily}

\usepackage{float}                      % De optie H voor de plaatsing van figuren op de plaats waar je ze invoegt. bvb. \begin{figure}[H]
%\usepackage{longtable}					% tabellen die over meerdere pagina's gespreid worden
%\usepackage[times]{quotchap}           % indien je fancy hoofdstuktitels wil
%\usepackage[none]{hyphenat}
%\usepackage{latexsym}
\usepackage{amsmath}
\usepackage{amssymb}

% MFA: zet zoekpad voor figure
\usepackage{subcaption}
\graphicspath{{fig/}}

\usepackage{fiiw_eng}

%door onderstaande regels in commentaar te zetten, of op false, kan je pagina's weglaten
%bijvoorbeeld het weglaten van een voorwoord, lijst met symbolen, ...
%%%%%%%%%%%%%%%%%%%%%%%%%%%%%%%%%%%%%%%%%%%%%%%%%%%%%%%%%%%%%%%%%%%%%%%%%%%%%%%%%%%%%%%%
%voorwoord toevoegen?
%\acknowledgementspagetrue
\acknowledgements{voorwoord}			%.tex file met daarin het voorwoord

%samenvatting toevoegen
%\summarypagetrue
\summary{samenvatting}					%.tex met daarin de samenvatting

%abstract toevoegen?
\abstractpagetrue
\abstracts{abstract}					%.tex file met daarin het abstract
%lijst van figuren toevoegen?
%\listoffigurespagetrue
%lijst van tabellen toevoegen?
% \listoftablespagetrue
%lijst van symbolen toevoegen?
%\listofsymbolspagetrue
%\listofsymbols{symbolen}				%.tex file met daarin de lijst van symbolen



%informatie over het eindwerk, de promotor, ...
%%%%%%%%%%%%%%%%%%%%%%%%%%%%%%%%%%%%%%%%%%%%%%%
\opleiding{Electronics-ICT}
\afdeling{}

\campus{denayereng} %denayer,denayereng,geel,geeleng,gent,ghenteng,groept,groupteng,brugge,brugeseng

\title{Inverting knowledge graphs back to raw data}
\subtitle{How can we leverage the rules we use to construct knowledge graphs to do the inverse?} 
% \author{naam student}
\forenameA{ }
\surnameA{ }

\forenameB{Tijs}
\surnameB{Van Kampen}

\academicyear{2023 - 2024}

\promotorA[Promotor]{Prof dr. ir. Anastasia Dimou}
\promotorB[Co-Promotor]{}



\begin{document}
%\selectlanguage{dutch} %due to incompatible syntax with the English style library and the extended features of the dutch library, I'll be modifying the dutch library to be English
\selectlanguage{english} % For the English version
\preface


%\addcontentsline{toc}{chapter}{List of abbreviations} % As this doesnt work, I'll leave it commented out for now
% \printglossary[title=List of abbreviations, type=\acronymtype]
%%%%%%%%%%%%%%%%%%%%%%%%%%%%%%%%%%%%%%%%%%%%%%%%%%%%%%%%%%%%%%%%%%% 
%                                                                 %
%                            CHAPTER                              %
%                                                                 %
%%%%%%%%%%%%%%%%%%%%%%%%%%%%%%%%%%%%%%%%%%%%%%%%%%%%%%%%%%%%%%%%%%% 

\chapter{Introduction}
\label{chapter:introduction}

The earliest academic definition of a knowledge graph can be found in a 1974 article as \begin{quote}
    A mathematical structure with vertices as knowledge units connected by edges that represent the prerequisite relation \citep{Marchi1974,bergman2019common}
\end{quote} 

The idea of expressing knowledge in a graph structure predates even this definition, with the concept of semantic networks \citep{Richens1956PreprogrammingFM}. % this is in the ago of punch card computers, so quite impressive 
However, the term knowledge graph only became well-known after Google announced they were using a knowledge graph to enhance their search engine in 2012 \citep{singhal2012introducing}. 
Knowledge graphs are used to make search engines, chatbots, question answering systems, etc more intelligent by injecting knowledge into them \citep{SurveyOnKGs}. 

A knowledge graph consists of many connected nodes, where each node is either an entity or a literal. These nodes are connected by edges, where each edge defines a relation between two nodes. \acrshort{rdf} is a framework often used to represent knowledge graphs, it uses subject-predicate-object triples to represent the nodes and their edges. Every node is either an \acrshort{uri}, a blank node or a literal. The edges are \acrshortpl{uri}. A triple representing the fact that the entity \texttt{John Doe} has the first name \texttt{John} would look like this: \texttt{http://example.com/John\_Doe http://schema.org/givenName "John"}. Often the predicates are chosen from an ontology/vocabulary, such as schema.org or \acrshort{foaf}. This allows for more interoperability between knowledge graphs, as the same predicates are used to represent the same concepts.

These knowledge graphs are constructed by extracting information from various sources, both unstructured sources such as text (using natural language processing) and (semi-)structured sources such as databases, CSV, XML, JSON, RDF (using mapping languages). Many mapping languages exist, differing on the way of defining the rules and the target source file format. Some mapping languages use the turtle syntax, while others provide their own custom syntax, and others repurpose existing languages like \acrshort{sparql} or \acrshort{shex}. \citep{VANASSCHE2023100753}. Some languages are specific to a single source format, such as R2RML(turtle format) \citep{Das:12:RRR} for relational databases, XSPARQL(\acrshort{sparql} format) \citep{Bischof2012} for XML. Others can process multiple formats, such as RML (turtle) \citep{dimou_ldow_2014}, D-REPR (\acrshort{yaml}), xR2RML (turtle), etc. These have the ability to map from multiple sources in different formats.

To achieve this these mapping languages use a declarative approach, where the user specifies the mapping rules, and the implementation of the mapping language takes care of the actual mapping. Two ways of doing the mapping exist: materialisation and virtualisation. Materialisation constructs the knowledge graph as a file, which can be loaded into a triple store. Virtualisation does not generate the knowledge graph as a file, but instead exposes a virtual knowledge graph, which can be queried as if it were a real knowledge graph. \citep{ontop}.

Creating these mapping rules is often done by hand. There are tools that make creating these mappings easier, like RMLEditor \citep{heyvaert_jws_2018} which exposes a visual editor and YARRRML \citep{10.1007/978-3-319-98192-5_40} which allows users to create rules in the user-friendly \acrshort{yaml} which are then compiled to RML rules. Alternatively tools are starting to be created for automatic generation of mapping rules from e.g. \acrshort{shacl}.
% TODO: do this citing

Retrieving data from a knowledge graph, for consumption by other programs, is done by querying the knowledge graph using SPARQL \citep{Seaborne:08:SQL} for tabular data and XSPARQL \citep{Bischof2012} or XSLT for XML. XSPARQL is the only language that can both map[/lift] and query[/lower], but the syntax for mapping and querying differs, so it could be argued that XSPARQL is actually two languages.

We can not convert the knowledge graph back to the original data format using the same rules we created it with. As such any changes we make to the data are hard to propagate back to the original data. We can not update, expand or improve the original data using e.g. knowledge graph refining. Nor can we apply changes to a virtual knowledge graph to change the original data. 

In thesis we seek to answer the question: \textit{How can we extend an existing system like RML or create a new system to construct
raw data from knowledge graphs?} We choose to extend the Morph-KGC implementation \citep{arenas2022morph} of \acrshort{rml} \citep{dimou_ldow_2014} as \acrshort{rml}'s end-to-end (from file to knowledge graph) characteristics make it a good candidate for this task. To answer the main research question we need to answer the following sub-questions:
\begin{itemize}
    \item[\textit{RQ1}] \textit{How can we construct the schema of the original data from the mapping rules?}
    \item[\textit{RQ2}] \textit{How can we populate the schema with data from the knowledge graph?}
\end{itemize}

\section{Thesis outline}
This thesis aims to explore the possibility of inverting knowledge graphs back to their original data format using RML mapping rules. To achieve this we will first look at the current state of the art in chapter \ref{chapter:related_work}. We will take a closer look at the technologies used like RDF, SPARQL, and RML. We will also look at the current state of the art for inverting knowledge graphs. In chapter \ref{chapter:implementation} we will look at our implementation of the inversion algorithm. We will look at the algorithm itself, and the implementation details. In chapter \ref{chapter:evaluation} we will evaluate our implementation using various benchmarks. For basic testing we use a subset of the rml test cases, which are designed to test the conformance of tools to the RML specification. For more advanced testing we will use various benchmarks simulating real-life use cases like LUBM4OBDA, GTFS-Madrid-Bench and SDM-Genomic-dataset. Finally in chapter \ref{chapter:conclusion} we will conclude this thesis, and look at possible future work.

%%%%%%%%%%%%%%%%%%%%%%%%%%%%%%%%%%%%%%%%%%%%%%%%%%%%%%%%%%%%%%%%%%% 
%                                                                 %
%                            CHAPTER                              %
%                                                                 %
%%%%%%%%%%%%%%%%%%%%%%%%%%%%%%%%%%%%%%%%%%%%%%%%%%%%%%%%%%%%%%%%%%% 

\chapter{Related work}
\label{chapter:related_work}

In this chapter, we discuss the various technologies related to this thesis. We begin by discussing the semantic web and build from there to technologies used within its ecosystem like \acrshort{rdf}, \acrshort{sparql}, and mapping languages. We finish by discussing the current state of the art in updating or creating the original data source from a knowledge graph.

\section{Semantic Web}
Tim Berneers-Lee envisioned a version of the web that would also be understandable by machines, and thus the semantic web was born. It is not designed as a separate entity to the web, but instead as an extension, mostly hidden for normal humans. It is designed mostly with existing technologies like \acrshort{xml}(including HTML, being a superset of it), \acrshort{uri} and \acrshort{rdf}. Even ontologies, a key component of the semantic web, are not a new concept but rather co-opted from the field of philosophy. \citep{thesemanticweb}

\section{RDF}
\acrshort{rdf} was originally designed as a data model for metadata but has since been extended to be a general-purpose framework for graph data. RDF is a directed graph, where the nodes represent entities, and the edges represent relations between these entities. This graph is built up from triples, which connect a subject and an object using a predicate as shown in figure \ref{fig:rdf_triple}.

\begin{figure}
    \centering
    \includegraphics[width=0.5\textwidth]{Chapter2/SPO}
    \caption{An RDF triple}
    \label{fig:rdf_triple}
\end{figure}

The subject must always be an entity, which can either be represented by an \acrshort{uri} or be a blank node. The predicate must be an \acrshort{uri}, and the object can be either an \acrshort{uri}, a blank node, or a literal. \citep{rdfprimer}

\begin{figure}[]
    \begin{tabular}{lll}
        \textbf{Subject}             & \textbf{Predicate}           & \textbf{Object}                           \\
        http://example.com/De\_Nayer & https://schema.org/location  & http://example.com/Sint\_Katelijne\_Waver \\
        http://example.com/r0785695  & https://schema.org/givenName & ``Tijs"
    \end{tabular}
    \caption{Example of RDF triples}
    \label{fig:rdf_triples_table}
\end{figure}

An \acrshort{uri} is a unique identifier for a resource on the web. Unlike a normal \acrshort{url}, it does not have to point to a network location, but can also be used to identify a person, a location, a concept, etc. \citep{rdfprimer}. In \acrshort{rdf} the \acrshort{uri} is purely used for identifying resources. As such, unlike in HTML where certain conventions are expected, there are no conventions for \acrshortpl{uri} in \acrshort{rdf}. An example of this can be seen in figure \ref{fig:rdf_triples_table}, the example subjects share the same domain but this does not imply that they are closely related, or even related at all. \acrshortpl{uri} are extended to \acrshortpl{iri} to allow for a wider range of characters. Except for allowing Unicode characters, \acrshortpl{iri} are identical to \acrshortpl{uri} so little distinction is made between the two in this thesis.

A blank node is a node that is not identified by a \acrshort{uri}. It is used to represent an anonymous resource that can't be or has no reason to be uniquely identified. For example, the address of student r0785695 in figure \ref{fig:blank_node} is only pertinent to the student and thus does not need to be uniquely identified. A blank node is serialized as \texttt{\_:name}, where name is a unique identifier for the blank node. This identifier is only unique within the document, and thus can't be used to refer to the blank node from outside the document. \citep{rdfprimer}

\begin{figure}[]
    \begin{tabular}{lll}
        \textbf{Subject}    & \textbf{Predicate}   & \textbf{Object}                                     \\
        ex:student/r0785695 & schema:address       & \_:addrr0785695                                     \\
        \_:addrr0785695     & schema:postalCode    & "2800"\textasciicircum \textasciicircum xsd:integer \\
        \_:addrr0785695     & schema:streetAddress & "Gentsesteenweg XXXX"@nl
    \end{tabular}
    \caption{Example of a blank node and prefix notation}
    \label{fig:blank_node}
\end{figure}

A literal is a value, e.g. a string, integer, or date. This value can be typed, e.g. a string can be typed as a date, or untyped. A string can also have a language tag, which is used to indicate the language of the literal.

RDF is only a framework, and as such does not define any serialization syntax. There are however a few common serialization standards for example RDF/XML, Turtle, N-Triples, and JSON-LD.

\subsection{Turtle}
Turtle, or Terse RDF Triple Language, is a human-readable serialization format for \acrshort{rdf}. It is the most used serialization format for \acrshort{rdf}, and is used in many tools and specifications.
In its simplest form turtle consists of triple statements, sequences of subject-predicate-object separated by spaces and terminated by a dot. An example of this can be seen in listing \ref{lst:basic_turtle_example}. This is very verbose, but Turtle offers many features to make it more concise. Below is a list of some of these features and, if possible, how they can be used to make the example more concise.
\begin{itemize}
    \item \textbf{Prefix notation} allows us to shorten \acrshortpl{uri} by defining a prefix.
          \begin{itemize}
              \item Using \texttt{@prefix schema: <https://schema.org/>} allows us to shorten \break\texttt{https://schema.org/Person} to \texttt{schema:Person}
          \end{itemize}
    \item \textbf{Base prefix} allows us to shorten \acrshortpl{uri} by defining a base \acrshort{uri}.
          \begin{itemize}
              \item Using \texttt{@base <http://example.com/>} allows us to shorten \break\texttt{http://example.com/r0785695} to \texttt{r0785695}
          \end{itemize}
    \item \textbf{Predicate lists} allow us to shorten multiple triples with the same subject to a list of predicates.
          \begin{itemize}
              \item Our example only has two subjects, we can split their predicates with \texttt{;} instead of repeating the subject.
          \end{itemize}
    \item \textbf{Object lists} allow us to shorten multiple triples with the same subject and predicate to a list of objects.
          \begin{itemize}
              \item \texttt{r0785695} is both a Person and a Student, so we can split the objects with \texttt{,} instead of repeating the subject and predicate.
          \end{itemize}
    \item \textbf{Literals} allow identifying values, e.g. strings, integers, dates, etc. with a datatype or language tag.
    \item \textbf{Blank nodes} allow us to define anonymous resources by using the \texttt{\_:} prefix.
          \begin{itemize}
              \item The address of \texttt{r0785695} is only relevant to \texttt{r0785695}, so we can define it as a blank node instead of using a \acrshort{uri}, this shortens \texttt{<http://example.com/addrr0785695>} to \texttt{\_:addrr0785695}. While not exactly the same, functionally it is equivalent as we do not expect addresses to be addressed outside of the context of a person.
          \end{itemize}
    \item \textbf{Unlabeled blank nodes} allow us to define anonymous resources without a unique identifier by using the [] notation instead of \texttt{\_:name}.
          \begin {itemize}
    \item As we do not need to refer to \texttt{\_:addrr0785695} from outside \texttt{r0785695}, we can use an unlabeled blank node and include it in \texttt{r0785695} instead of a labeled blank node.
\end{itemize}
\item \textbf{Collections} allow us to define a list of blank nodes by using the () notation.
\end{itemize}
The example in listing \ref{lst:basic_turtle_example} can be rewritten using these features, as shown in listing \ref{lst:basic_turtle_example_concise}.


\begin{lstlisting}[caption={Basic naive turtle document}, label={lst:basic_turtle_example}, captionpos=b, breaklines=true, basicstyle=\small]
    <http://example.com/r0785695> <http://www.w3.org/1999/02/22-rdf-syntax-ns#type> <http://schema.org/Person> .
    <http://example.com/r0785695> <http://www.w3.org/1999/02/22-rdf-syntax-ns#type> <http://schema.org/Student> .
    <http://example.com/r0785695> <http://schema.org/givenName> "Tijs" .
    <http://example.com/r0785695> <http://schema.org/familyName> "Van Kampen" .
    <http://example.com/r0785695> <http://schema.org/address> <http://example.com/addrr0785695> .
    <http://example.com/addrr0785695> <http://schema.org/postalCode> "2800"^^<http://www.w3.org/2001/XMLSchema#integer> .
    <http://example.com/addrr0785695> <http://www.w3.org/1999/02/22-rdf-syntax-ns#type> <http://schema.org/PostalAddress> .
    <http://example.com/addrr0785695> <http://schema.org/streetAddress> "Gentsesteenweg XXXX"@nl .
    <http://example.com/addrr0785695> <http://schema.org/addressCountry> "Belgium" .
\end{lstlisting}

\begin{lstlisting}[caption={Basic turtle document using turtle features}, label={lst:basic_turtle_example_concise}, captionpos=b, breaklines=true]
    @prefix schema: <https://schema.org/> .
    @prefix xsd: <http://www.w3.org/2001/XMLSchema#> .
    @base <http://example.com/> .
    <r0785695> a schema:Person, schema:Student ;
        schema:givenName "Tijs" ;
        schema:familyName "Van Kampen" ;
        schema:address [
            a schema:PostalAddress ;
            schema:postalCode "2800"^^xsd:integer ;
            schema:streetAddress "Gentsesteenweg XXXX"@nl ;
            schema:addressCountry "Belgium"
        ] .
\end{lstlisting}


\section{SPARQL}
\acrfull{sparql} is the \acrshort{w3c} standard query language for \acrshort{rdf}. It is the main way to query \acrshort{rdf} data and shows many similarities to SQL. \acrshort{sparql} queries mostly consist of a pattern of triples, which are matched against the \acrshort{rdf} graph, a basic example can be found in listing \ref{lst:sparql_select_query}. Querying is very feature-rich, with support for aggregation, subqueries, negation, regex, string manipulation, etc. It also supports different return types, federated queries, entailment, etc \citep{SPARQL1.1QL}. Aside from the query protocol it also defines the graph store protocol, which can be used to manipulate graph databases directly \citep{SPARQL1.1}.

\begin{lstlisting}[language=SPARQL, caption={Example of a basic \acrshort{sparql} SELECT query}, label={lst:sparql_select_query}, captionpos=b]
    PREFIX schema: <https://schema.org/>
    SELECT ?name ?address
    WHERE {
        ?student schema:givenName ?name .
        ?student schema:address ?address .
    }
\end{lstlisting}

\section{Mapping languages}
Mapping languages are used to define a mapping between a source and a target. The target in the context of linked data is of course \acrshort{rdf}, with the source being any structured data source. Some mapping languages exist for a single source, e.g. \acrfull{r2rml} for relational databases, \acrfull{xsparql} for \acrshort{xml}, etc. Others are more general purpose, e.g. \acrfull{rml} and \acrfull{d2rml}.
We will discuss both \acrshort{r2rml} and \acrshort{rml} in more detail, as one extends the other. \acrshort{rml} we will discuss because it is one of the more feature-rich general mapping languages, and it is the mapping language we will use in our implementation. \acrshort{r2rml} we discuss because it is the most widely used mapping language, as it supports virtualization \textit{ontop} of databases. % ontop is a tool that allows virtualization
Most \acrshort{rml} implementations also support \acrshort{r2rml}, as \acrshort{rml} is a nearly superset of \acrshort{r2rml}.

\subsection{\acrshort{r2rml}}
\acrfull{r2rml} is a mapping language for mapping relational databases to \acrshort{rdf}. As opposed to \acrfull{dm}, which results in a direct mapping from the relational database to \acrshort{rdf} without any changes to structure or naming, \acrshort{r2rml} allows for more flexibility. R2RML mappings consist of zero or more TriplesMaps, which are used to map a table to \acrshort{rdf}. A TriplesMap consists of a logical table, a subject map, and one or more \acrfullpl{pom}.

The logical table is used to define the table that is being mapped, with each row in the table being mapped to a subject and its corresponding \acrshortpl{pom}. It is possible to create a view of a table by using a SQL query, and then map this view. This allows for more complex mappings, e.g. mapping a join of two tables or a computed column.
% Could add this, but as we don't actually support R2RML in our implementation for now this is not really relevant
% An example of this can be seen in listing \ref{lst:r2rml_mapping_view}.

% \begin{lstlisting}[caption={Example of an \acrshort{r2rml} mapping with a view. Snippet from \citep{r2rml}}, label={lst:r2rml_mapping_view}, captionpos=b] 
%     <#TriplesMap1>
%     rr:logicalTable [ rr:sqlQuery """
%         SELECT EMP.*, (CASE JOB
%             WHEN 'CLERK' THEN 'general-office'
%             WHEN 'NIGHTGUARD' THEN 'security'
%             WHEN 'ENGINEER' THEN 'engineering'
%         END) ROLE FROM EMP
%         """ ];
%     rr:subjectMap [
%         rr:template "http://data.example.com/employee/{EMPNO}";
%     ];
%     rr:predicateObjectMap [
%         rr:predicate ex:role;
%         rr:objectMap [ rr:template "http://data.example.com/roles/{ROLE}" ];
%     ].
% \end{lstlisting}

Each of the SubjectMap, PredicateMap, ObjectMap, (and GraphMap) is a subclass of TermMap, which is a function that generates an \acrshort{rdf} term. The map type can be constant, template, or column. The resulting term is then used as the subject, predicate, object, or graph of the triple. The termType of the map determines the type of the term, which can be \acrshort{iri}, blank node, or literal. If the termType is literal, optionally the datatype or language can be added. Following the \acrshort{rdf} specification, not all combinations of termType and map are possible, this is shown in table \ref{tab:termType_map_combinations}. The object map has an additional subclass, a reference object map, in which we refer to another TriplesMap. Using a reference map we can map a foreign key to the subject of another TriplesMap, with a join condition. \citep{r2rml}

\begin{table}[h]
    \begin{tabular}{|l|l|l|l|l|}
        \hline
        \textbf{TermType} & \textbf{Subject} & \textbf{Predicate} & \textbf{Object} & \textbf{Graph} \\ \hline
        IRI               & \cmark           & \cmark             & \cmark          & \cmark         \\ \hline
        Blank node        & \cmark           & \xmark             & \cmark          & \cmark         \\ \hline
        Literal           & \xmark           & \xmark             & \cmark          & \xmark         \\ \hline
    \end{tabular}
    \caption{Possible combinations of TermType and Map type}
    \label{tab:termType_map_combinations}
\end{table}

A constant value is a fixed value, e.g. a \acrshort{uri} or a string. A template is a string with placeholders, which are replaced by values from the logical row. A column is the value of a column in the logical row.

\begin{lstlisting}[language=XML, caption={Example of an \acrshort{r2rml} mapping}, label={lst:r2rml_mapping}, captionpos=b]
@prefix rr: <http://www.w3.org/ns/r2rml#>.
@prefix ex: <http://example.com/ns#>.

<#TriplesMap1>
    rr:logicalTable [ rr:tableName "EMP" ];
    rr:subjectMap [
        rr:template "http://data.example.com/employee/{EMPNO}";
        rr:class ex:Employee;
    ];
    rr:predicateObjectMap [
        rr:predicate ex:id;
        rr:objectMap [ rr:column "EMPNO"; rr:datatype xsd:positiveInteger ].
    ].
\end{lstlisting}

\subsection{\acrshort{rml}}
\acrfull{rml} is a mapping language for mapping any (semi-)structured data source to \acrshort{rdf}. It is a generalization of \acrshort{r2rml} and as such supports all the features of \acrshort{r2rml}. It extends \acrshort{r2rml} by extending database-specific features to make them more general. The differences in usage can be seen in table \ref{tab:r2rml_rml_differences}. \citep{rml}

\begin{table}[]
    \begin{tabular}{|ll|ll|}
        \hline
        \multicolumn{2}{|c|}{R2RML}         & \multicolumn{2}{c|}{RML}                                                           \\ \hline
        Logical Table (relational database) & rr:logicalTable          & Logical Source               & rml:logicalSource        \\ \hline
        Table Name                          & rr:tablename             & URI (pointing to the source) & rml:source               \\ \hline
        column                              & rr:column                & reference                    & rml:reference            \\ \hline
        (SQL)                               & rr:SQLQuery              & Reference Formulation        & rml:referenceFormulation \\ \hline
        per row iteration                   &                          & defined iterator             & rml:iterator             \\ \hline
    \end{tabular}
    \caption{Differences between \acrshort{r2rml} and \acrshort{rml}}
    \label{tab:r2rml_rml_differences}
\end{table}

\acrshort{rml} uses the same structure as \acrshort{r2rml}, with TriplesMaps consisting of a logical source, a subject map, and zero or more \acrshortpl{pom}. The changes it has all relate to the logical source. Whereas in \acrshort{r2rml} the source is always a database, from which we select a table or view, in \acrshort{rml} the source can be one of many different source types like \acrshort{xml}, \acrshort{json}, \acrshort{csv}, etc. Where in \acrshort{r2rml} we simply iterate over the rows of a table, in \acrshort{rml} we can have a source without an explicit iteration pattern, and as such we need to define an iterator.

% \section{if meaningful: provenance}
% provenance could be interesting but is not mainstream enough to use on a larger scale. (-yet? Maybe in future work add a reference)

\section{State of the art}
The state of the art in updating or creating the data source from a knowledge graph can be split in two categories, depending on the methodology used. The first methodology applies to virtualization, where the data is exposed as a virtual knowledge graph over the source data. The other methodology is for materialized knowledge graphs, where the knowledge graph is created as a file that can be loaded into a triple store. We will discuss the state of the art in both methodologies, concluding with the relevance of this work.

\subsection{Virtualization}
Virtualization is the process of exposing a virtual knowledge graph over the source data. This virtual knowledge graph can be queried as if it were a real knowledge graph. To achieve this mappings are used to translate queries over the knowledge graph to queries over the source data.

This is most commonly used to expose a database as a virtual knowledge graph. This way an organization can expose a knowledge graph without having to completely transition to a new system. Most implementations of virtualization layers are read-only though, as the translation of SELECT queries is relatively easy but translating INSERT, DELETE or DELETE/INSERT (update) queries is much less straightforward, or even impossible in many cases. Though propagating changes trough the virtualization layer to the source data could be a big part of the linked data lifecycle, related work on this is scarce. In both "SPARQL Update queries over R2RML mapped data sources" \citep{unbehauen-k-2017--sparqlUpdate} and "Practical Update Management in Ontology-Based Data Access" \citep{practical_update_management_in_ontology_based_data_access} the authors propose a similar way of handling updates. Compatible updates are propagated to the source data, while incompatible updates are held in an 'overflow' triple store. Further changes may make the incompatible updates compatible, at which point they too are propagated to the source data. Larger changes affecting the general structure of the data are not stored, but instead the mapping is updated to reflect the new structure. The handling of updates is shown in figure \ref{fig:virtualization_update}.

\begin{figure}
    \centering
    \includegraphics[width=0.6\textwidth]{Chapter2/Write-also_ODBA_architecture.png}
    \caption{Propagating changes in a virtualization layer}
    \label{fig:virtualization_update}
\end{figure}

\subsection{Materialization}
Materialization is the process of creating a knowledge graph as a file that can be loaded into a triple store. The knowledge graph is then loaded into a dedicated triple store for querying. This method benefits from increased performance, at the cost of having the knowledge graph not in sync with the source data.

Materialization allows for a wider range of source formats, as the knowledge graph can be created from any structured data source. This is especially useful when the source data is not easily queryable, e.g. when the source data is a \acrshort{csv}, \acrshort{xml}, or \acrshort{json} file. For knowledge graphs created from structured data sources, use cases exist for propagating changes back to the original data source. This is not done using a direct update (as the source data is not directly connected to the knowledge graph), but by creating a new version of the source data. This process is called lowering. Below we discuss some of the state of the art in lowering.

\subsubsection{\acrshort*{xsparql}}
\acrshort{xsparql} is a mapping language that allows for the transformation of \acrshort{xml} to and back from \acrshort{rdf}. It expands XQUERY with SPARQL-like syntax, structure and features. Its predecessors would do lifting by querying the source \acrshort{xml} using \acrshort{xquery} to transform it into the \acrshort{xml} serialization of \acrshort{rdf}, while lowering was done using XSLT to do the inverse. Using its combined vocabulary \acrshort{xsparql} simplifies lifting and lowering, using a single language for both and having the ability to target the turtle serialization \citep{xsparql}. An example lifting and lowering query can be found in listing \ref{lst:xsparql_lifting} and listing \ref{lst:xsparql_lowering} respectively.

\begin{lstlisting}[caption={Example of \acrshort{xsparql} lifting}, label={lst:xsparql_lifting}, captionpos=b, basicstyle=\small]
declare namespace foaf="http://xmlns.com/foaf/0.1/";
declare namespace rdf="http://www.w3.org/1999/02/22-rdf-syntax-ns#";
let $persons := //*[@name or ../knows]
return

for $p in $persons
let $n := if( $p[@name] )
            then $p/@name else $p
let $id := count($p/preceding::*)
            +count($p/ancestor::*)
where
    not(exists($p/following::*[
        @name=$n or data(.)=$n]))
construct {
    [*_:b{$id} a foaf:Person;*]
                [*foaf:name {data($n)}.*]
    {
        for $k in $persons
        let $kn := if( $k[@name] )
                    then $k/@name else $k
        let $kid := count($k/preceding::*)
                    +count($k/ancestor::*)
        where
            $kn = data(//*[@name=$n]/knows) and
            not(exists($kn/../following::*[
                @name=$kn or data(.)=$kn]))
        construct {
        [*_:b{$id} foaf:knows _:b{$kid}.*]
        [*_:b{$kid} a foaf:Person.*]
        }
    }
}
\end{lstlisting}

\begin{lstlisting}[caption={Example of \acrshort{xsparql} lowering}, label={lst:xsparql_lowering}, captionpos=b, basicstyle=\small]
<relations>{
    for $Person $Name from <relations.rdf>
    where {$Person foaf:name $Name}
    order by $Name
    return
        <person name="{$Name}">{
            for $FName from <relations.rdf>
            where {
                $Person foaf:knows $Friend.
                $Person foaf:name $Name.
                $Friend foaf:name $Fname. }
            return
            <knows>{$FName}</knows>
        }</person>
}</relations>
\end{lstlisting}


\subsubsection{POSER: A Semantic Payload Lowering Service \citep{poser}}
POSER(Payload lOwering SERvice) is a service that lowers \acrshort{rdf} to \acrshort{json}. To achieve this a mapping is created in two parts: first the source patterns are defined from which to extract the data, then the json structure is defined. The mapping is written in turtle, using a custom json ontology describing the json structure. A proof of concept implementation was made, but never got out of the prototype phase. It also only handles direct literal types, more complex composite values that are generated from templates are not supported. An example mapping is shown in listing \ref{lst:poser_mapping}.

\begin{lstlisting}[caption={Example of a POSER mapping}, label={lst:poser_mapping}, captionpos=b, basicstyle=\small]
@prefix ctd: <http://connectd.api/> .
@prefix onto: <http://ontodm.com/OntoDT#> .
@prefix iots: <http://iotschema.org/> .
@prefix json: <http://some.json.ontology/> .
@prefix xsd: <http://www.w3.org/2001/XMLSchema#> .
@prefix time: <https://www.w3.org/TR/2020/CR-owl-time-20200326/> .

# Which inputs to expect and to start mapping from
json:InputDataType {
    json:EntryPoint a iots:TimeSeries;
        iots:providesTemperatureData iots:TemperatureData;
        iots:providesTimeData iots:TimeData .

    iots:TemperatureData iots:numberDataType iots:Number .
    iots:TimeData  time:dateTime iots:Number .
}

#Semantic description of the json objects to be found in the expected API

json:ApiDescription {
    ctd:JsonModel json:hasRoot ctd:Node .

    ctd:TemperatureValue a json:Number ;
        json:key "value"^^xsd:string ;
        json:dataType iots:TemperatureData .

    ctd:TimeStamp a json:String ;
        json:key "timestamp"^^xsd:string ;
        json:dataType iots:TimeData .

    ctd:Node a json:Object;
        json:key "node"^^xsd:string ;
        json:value ctd:TimeStamp, ctd:TemperatureValue ;
        json:dataType iots:TimeSeries .

    ctd:Edges a json:Array	;
        json:key "edges"^^xsd:string ;
        json:value ctd:Node .
}
\end{lstlisting}

\subsection{Relevance}
As shown in this section the state of the art in updating the source in virtualization is pretty mature, only limited by fundamental limitations. It is however limited to databases. To work with other (semi-)structured data sources materialization is needed. For materialization the state of the art is much more limited. Though methods exist to lower \acrshort{rdf} to other formats, each method is intimately linked to a source type. The mappings are also unidirectional, even \acrshort{xsparql} which offers both lifting and lowering requires a separate mapping for each. 

Our work expands RML, which can map from any structured data source to \acrshort{rdf}, with the ability to lower the \acrshort{rdf} back to the original source. We use a single mapping to do both lifting and lowering, as information on where to find the data during lifting can also be used to find where to put the data during lowering. This makes our work unique in the field of lowering \acrshort{rdf} to other formats.
%%%%%%%%%%%%%%%%%%%%%%%%%%%%%%%%%%%%%%%%%%%%%%%%%%%%%%%%%%%%%%%%%%% 
%                                                                 %
%                            CHAPTER                              %
%                                                                 %
%%%%%%%%%%%%%%%%%%%%%%%%%%%%%%%%%%%%%%%%%%%%%%%%%%%%%%%%%%%%%%%%%%% 
\chapter{Implementation}
\label{chapter:implementation}
This chapter is split into three sections. In the first section the structure of the algorithm is outlined. In the second and third sections, we go into detail about the two sub-research questions: how to construct the schema and how to populate the schema. 


% At this time we only have a PoC implementation, as such many details haven't been worked out yet. For example, the section on creating the schema is limited, as we only handle CSV files which require little to no schema. 

For the implementation, we assume that the mapping rules never map to a superset of the knowledge graph. 

The implementation is written in Python. As we seek to extend Morph-KGC we make use of its internal functions and the libraries to work with those, like pandas. Querying the knowledge graph is done with the SPARQLWrapper library.

\section{General structure}
The algorithm consists of 4 main parts: setting up, creating the schema, retrieving the data, and applying the data to the schema. Creating the schema and retrieving the data are the two main parts of the implementation, and are covered in their sections. Setting up consists of setting up the \acrshort{sparql} endpoint if necessary and processing the mapping files. Applying the data to the schema consists of applying the data of each iteration to the schema, and creating the output file. A graphical representation of the algorithm can be found in figure \ref{fig:algorithm}. A more formal representation can be found in algorithm \ref{alg:inversion}. The functions used in the algorithm are described in the relevant sections.

\begin{figure}[h]
    \centering
    \includegraphics[width=\textwidth]{Chapter3/algorithm_flat.png}
    \caption{Simplified overview of the algorithm}
    \label{fig:algorithm}
\end{figure}

\begin{algorithm}
    \caption{Inversion algorithm}
    \label{alg:inversion}
    \begin{algorithmic}[1]
        \Require{$morph\_config$ is a morph-kgc config file describing the mapping files and output location}
        \State $config \gets load_config(morph\_config)$
        \State $mapping\_rules \gets retrieve\_mappings(config)$
        \State $mapping\_rules \gets add\_helper\_columns(mapping\_rules)$
        \State $graph\_location \gets config.get\_output\_file()$
        \State $graph\_endpoint \gets load\_graph(graph\_location)$
        \ForAll{$source, source\_rules \in mapping\_rules.groupby('source')$}
            \ForAll{$iterator, iterator\_rules \in source\_rules.groupby('iterator')$}
                \State $query \gets generate\_query(iterator\_rules)$
                \State $values[iterator] \gets graph.query(query)$
            \EndFor
            \State $templates \gets generate\_templates(source\_rules)$
            \State $source\_output \gets apply\_templates(templates, values)$
        \EndFor
    \end{algorithmic}
\end{algorithm}

% I made this as a joke, this is in no way readable so the algorithm is split over the sections
% \begin{algorithm}
%     \caption{Full inversion algorithm}
%     \label{alg:inversion}
%     \begin{algorithmic}[1]
%         \Require{$morph\_config$ is a morph-kgc config file describing the mapping files and output location}
%         \State $config \gets load_config(morph\_config)$
%         \State $mapping\_rules \gets retrieve\_mappings(config)$
%         \State $mapping\_rules \gets add\_helper\_columns(mapping\_rules)$
%         \State $graph\_location \gets config.get_output_file()$
%         \State $graph\_endpoint \gets load\_graph(graph\_location)$
%         \ForAll{$source, source\_rules \in mapping\_rules.groupby('source')$}
%             \ForAll{$iterator, iterator\_rules \in source\_rules.groupby('iterator')$}
%                 \ForAll{$subject, subject\_rules \in iterator\_rules.groupby('subject')$}
%                     \ForAll{$rule \in subject\_rules$}
%                         \If{$rule.is\_constant()$}
%                             \State $query\_lines.append(rule.to\_triple())$ 
%                         \ElsIf{$rule.is\_reference()$}
%                             \State $query\_lines.append(rule.to\_triple())$
%                         \ElsIf{$rule.is\_template()$}
%                             \State $query\_lines.append(test\_object\_regex(rule))$
%                             \State $remainder \gets rule['object\_map\_value']$
%                             \ForAll{$reference \in rule.get\_references()$}
%                                 \If{$remainder.empty()$}
%                                     \State $query\_lines.append(bind\_str\_after\_ref(reference)) $
%                                 \Else
%                                     \State $query\_lines.append(bind\_str\_after\_remainder()) $
%                                     \State $query\_lines.append(bind\_str\_before(reference)) $
%                                 \EndIf
%                                 \State $remainder \gets remainder - reference$
%                             \EndFor
%                         \EndIf
%                         \State $query \gets wrap\_query\_lines(query\_lines)$
%                     \EndFor                                
%                 \State $query \gets generate\_query(iterator\_rules)$
%                 \State $values[iterator] \gets graph\_endpoint.query(query)$
%                 \EndFor
%             \EndFor
%             \If{$source\_rules['source\_type'] == 'CSV'$}
%                 \State $source\_output \gets values[0]$
%             \EndIf
%         \EndFor
%     \end{algorithmic}
% \end{algorithm}

\subsection{Loading the mapping rules}
We start by loading the mapping rules from the mapping files. We use Morph-KGC's internal \texttt{retrieve\_mappings} function for this. This function takes a loaded config file (.ini format) as input, which specifies the location of the mapping files and various other settings for which we have little use as they are only used for the materialization process. When relevant later on we could add our settings to this config file.
The mappings are returned as a pandas DataFrame. As the pandas library has the functionality to group by columns, we can easily group the mapping rules by source and iterator later. An example of a single mapping rule (row in the DataFrame) can be found in listing \ref{lst:mapping_rule}. Marked in bold are the helper columns we add.

\begin{lstlisting}[caption={Example of a mapping rule in Morph-KGC}, label={lst:mapping_rule}, captionpos=b, basicstyle=\small]
source_name: DataSource1
triples_map_id: #TM0
triples_map_type: http://w3id.org/rml/TriplesMap
logical_source_type: http://w3id.org/rml/source
logical_source_value: student.csv
iterator: nan
subject_map_type: http://w3id.org/rml/template
subject_map_value: http://example.com/{Name}
[*subject_references_template: http://example.com/([^\/]*)$*]
[*subject_references: ['Name']*]
[*subject_reference_count: 1*]
subject_termtype: http://w3id.org/rml/IRI
predicate_map_type: http://w3id.org/rml/constant
predicate_map_value: http://xmlns.com/foaf/0.1/name
[*predicate_references_template: None
predicate_references: []
predicate_reference_count: 0*]
object_map_type: http://w3id.org/rml/reference
object_map_value: Name
[*object_references_template: None
object_references: ['Name']
object_reference_count: 1*]
object_termtype: http://w3id.org/rml/Literal
object_datatype: nan
object_language: nan
graph_map_type: http://w3id.org/rml/constant
graph_map_value: http://w3id.org/rml/defaultGraph
subject_join_conditions: nan
object_join_conditions: nan
source_type: CSV
mapping_partition: 1-1-1-1
\end{lstlisting}

\section{Contructing the schema}
\label{section:constructing_schema}
Constructing the schema is done by reversing the mapping rules' source. We do this using the iterator and the mapping rule's references. \acrshort{rml} supports many different types of sources and referenceFormulations. We will implement the CSV, xPath, and JSONPath referenceFormulations. Not every source is a file, so for query-based sources, we will generate the query output. In a later stage, we could look into taking it a step further, generating the actual source behind those intermediate results.

As each referenceFormulation has its reference syntax, we will have to tailor the implementation to each referenceFormulation. For the PoC, we only implement the CSV referenceFormulation. 

\subsection{CSV}
\label{subsection:csv}
The CSV referenceFormulation is the simplest of the three as it describes a simple two-dimensional table, with columns having the names of the references and rows being the iterated values. The example TriplesMap in listing \ref{lst:csv_file_mapping} results in the CSV template in listing \ref{lst:csv_file}. Unlike the other referenceFormulations, CSV has no uncertainty in terms of structure.

\begin{lstlisting}[caption={Example mapping for a CSV file}, label={lst:csv_file_mapping}, captionpos=b, basicstyle=\small]
<TriplesMap1> a rr:TriplesMap;

rml:logicalSource [ 
    rml:source "student.csv";
    rml:referenceFormulation ql:CSV
];

rr:subjectMap [ 
    rr:template "http://example.com/Student/{ID}/{Name}";
    rr:graph ex:PersonGraph ;
    rr:class foaf:Person
];

rr:predicateObjectMap [ 
    rr:predicate ex:id ; 
    rr:objectMap [ rml:reference "ID" ]
];

rr:predicateObjectMap [ 
    rr:predicate foaf:name ; 
    rr:objectMap [ rml:reference "Name" ]
].
\end{lstlisting}

\begin{lstlisting}[caption={Example CSV template}, label={lst:csv_file}, captionpos=b, basicstyle=\small]
[*ID,Name*]
<ID>,<Name>
\end{lstlisting}

\section{Retrieving the data}
\label{section:retrieving_data}
Retrieving the data is done by querying the knowledge graph. We do this by generating SPARQL queries matching the mapping rules. We then execute these queries and store the results. Mappings sharing the same iterator are executed together, as they have been generated in the same iteration. 

\subsection{Generating the queries}
\label{subsection:generating_queries}
We generate the queries by translating the mapping rules into triple patterns. 
For each of the three map types, constant, reference, and template, we adapt our approach. 
For the constant map type, we know that it will always be present with a constant value, so we can simply add it as a triple pattern like \texttt{?s a foaf:Person .}. 
Both reference and template map types will not generate a triple during materialization if any of the references are not present during the iteration. As such, they are wrapped in an optional block. 
Reference maps are the easiest to work with as they directly translate back to the source. There is an exception to this, as a reference map can be used for a subject map. In this case, the value is either an \acrshort{uri} or appended to the base \acrshort{uri}. Sadly Morph-KGC does not support the latter, so we can not take this into account. An example of a basic query using constant and reference maps can be found in listing \ref{lst:simple_query_example}.

\begin{lstlisting}[caption={Simple query example}, label={lst:simple_query_example}, captionpos=b]
SELECT DISTINCT ?Name ?ID
WHERE {
    ?s a foaf:Person .

    optional{
        ?s foaf:name ?Name .
    }
    optional{
        ?s ex:id ?ID .
    }
}
\end{lstlisting}    

Template maps are the most complex to work with as their structure can wildly vary. The simplest step we can take is to confirm the mapped value matches with the map template like
\texttt{FILTER(regex(str(?s), "http://example.com/([\textasciicircum\textbackslash\textbackslash/]*)/([\textasciicircum\textbackslash\textbackslash/]*)\$"))}. Taking this further we can use string manipulation to split the variable into the different references. An example of this can be found in listing \ref{lst:template_query_example}.

\begin{lstlisting}[caption={Template query example}, label={lst:template_query_example}, captionpos=b]
SELECT DISTINCT ?Name ?ID
WHERE {
    ?s a foaf:Person .
    FILTER(regex(str(?s), "http://example\\.com/([^\\/]*)/([^\\/]*)$")) .
    BIND(STRAFTER(str(?s), "http://example.com/") as ?temp) .
    BIND(STRBEFORE(str(?temp), "/") as ?ID)
    BIND(STRAFTER(?temp, "/") as ?Name)

    optional{
        ?s foaf:name ?Name .
    }
    optional{
        ?s ex:id ?ID .
    }
}
\end{lstlisting}

The algorithm for generating the queries can be found in algorithm \ref{alg:generate_query}. This is still an early version of the algorithm, which needs to be improved to handle more complex mappings.

\begin{algorithm} 
    \caption{Generating the queries}
    \label{alg:generate_query}
    \begin{algorithmic}[1]
        \Require{$iterator$ is the iterator to generate the query for}
        \Require{$mapping\_rules$ is a set of mapping rules for said iterator}
        \State $query\_lines \gets []$
        \ForAll{$rule \in mapping\_rules$}
            \If{$rule.is\_constant()$}
                \State $query\_lines.append(rule.to\_triple())$ 
            \ElsIf{$rule.is\_reference()$}
                \State $query\_lines.append(rule.to\_optional\_triple())$
            \ElsIf{$rule.is\_template()$}
                \State $query\_lines.append(test\_object\_regex(rule))$
                \State $remainder \gets rule['object\_map\_value']$
                \ForAll{$reference \in rule['object\_references']$}
                    \State $query\_lines.append(bind_reference_part(rule, reference))$
                \EndFor
            \EndIf
        \EndFor
        \State $query \gets wrap\_query\_lines(query\_lines)$
    \end{algorithmic}
\end{algorithm}

\subsection{Disjointed mappings}

We generate a single query for each iterator. This query can contain multiple subjects. This can, however, lead to issues when the subjects share no references. The effect we get is not unlike joining two tables in SQL without join conditions. For example, using the mapping listed in listing \ref{lst:bad_join_example} we get the query in listing \ref{lst:bad_join_query}. When applied to the knowledge graph in listing \ref{lst:bad_join_kg} we get the result in listing \ref{lst:bad_join_result} instead of the original source in listing \ref{lst:bad_join_expected_result}. When converting the badly generated source back to the knowledge graph, we do get the same knowledge graph as the original as the duplicate data is ignored. The amount of duplicate data increases exponentially with the number of subjects so even though ignoring it would be a valid solution, it is not viable with larger datasets. The only solution to this problem is updating the mapping rules to either split the source or add shared references. The user is ultimately responsible for this, but we could generate a warning to notify the user. 

\begin{listing}
    \refstepcounter{lstlisting}
    \noindent\begin{minipage}[b]{.45\textwidth}
        \begin{lstlisting}[basicstyle=\small]
<TriplesMap1> a rr:TriplesMap;
rml:logicalSource [ 
    rml:source "student_sport.csv";
    rml:referenceFormulation ql:CSV
];
rr:subjectMap [ 
    rr:template "http://example.com/{Student}";
    rr:class ex:Student
];
rr:predicateObjectMap [ 
    rr:predicate foaf:name ; 
    rr:objectMap [ 
        rml:reference "Student"
    ]
].
        \end{lstlisting}      
    \end{minipage}
    \hfill
    \begin{minipage}[b]{.45\textwidth}
        \begin{lstlisting}[basicstyle=\small]
<TriplesMap2> a rr:TriplesMap;
rml:logicalSource [ 
    rml:source "student_sport.csv";
    rml:referenceFormulation ql:CSV
];
rr:subjectMap [ 
    rr:template "http://example.com/{Sport}";
    rr:class ex:Sport
];
rr:predicateObjectMap [ 
    rr:predicate foaf:name ; 
    rr:objectMap [ 
        rml:reference "Sport"
    ]
].
        \end{lstlisting}
    \end{minipage}
    \addtocounter{listing}{5}
    \caption{Bad join mapping}
    \label{lst:bad_join_example}
\end{listing}

\begin{lstlisting}[caption={Bad join query (trimmed)}, label={lst:bad_join_query}, captionpos=b, basicstyle=\small]
SELECT DISTINCT ?Student_name ?Sport
WHERE {
    ?s1 a ex:Student .
    optional{
        ?s1 foaf:name ?Student_name .
    }

    ?s2 a ex:Sport .
    optional{
        ?s2 foaf:name ?Sport .
    }
}
\end{lstlisting}

\begin{lstlisting}[caption={Bad join knowledge graph}, label={lst:bad_join_kg}, captionpos=b, basicstyle=\small]
@prefix ex: <http://example.com/> .
@prefix foaf: <http://xmlns.com/foaf/0.1/> .

ex:Venus a ex:Student ;
    foaf:name "Venus" .
ex:Tom a ex:Student ;
    foaf:name "Tom" .
ex:Tennis a ex:Sport ;
    foaf:name "Tennis" .
ex:Football a ex:Sport ;
    foaf:name "Football" .
\end{lstlisting}

\begin{lstlisting}[caption={Bad join result}, label={lst:bad_join_result}, captionpos=b, basicstyle=\small]
Student,Sport
Venus,Tennis
Venus,Football
Tom,Tennis
Tom,Football
\end{lstlisting}

\begin{lstlisting}[caption={Bad join original source}, label={lst:bad_join_expected_result}, captionpos=b]
Student,Sport
Venus,Tennis
Tom,Football
\end{lstlisting}

\section{Applying the data to the schema}
\label{section:applying_data}
Generating the final output is done by iterating over the rows of the data and applying them to the schema. Each column of the data corresponds to a reference in the schema. A short version of the algorithm can be found in algorithm \ref{alg:apply_data}. For the PoC this approach is sufficient, even too complex as we can simply dump the query-result DataFrame to a CSV file. For more complex, possibly nested, sources we will have to adapt this algorithm.

\begin{algorithm}
    \caption{Applying the data to a simple (non-nested) schema}
    \label{alg:apply_data}
    \begin{algorithmic}[1]
        \Require{$schema$ is a schema}
        \Require{$data$ is a DataFrame}
        \State $output \gets new\_file$
        \ForAll{$row \in data$}
            \State $output \gets schema$
            \ForAll{$column \in row$}
                \State $schema.replace(column.name, column.value)$
            \EndFor
            \State $output.write(schema)$
        \EndFor
    \end{algorithmic}
\end{algorithm}
%%%%%%%%%%%%%%%%%%%%%%%%%%%%%%%%%%%%%%%%%%%%%%%%%%%%%%%%%%%%%%%%%%% 
%                                                                 %
%                            CHAPTER                              %
%                                                                 %
%%%%%%%%%%%%%%%%%%%%%%%%%%%%%%%%%%%%%%%%%%%%%%%%%%%%%%%%%%%%%%%%%%% 
\chapter{Evaluation}
\label{chapter:evaluation}
In this chapter we will evaluate our implementation. We will do this by testing it again various datasets, comparing the expected results with the actual results. For each dataset we will go over our testing methodology and its results.

\section{RML test cases}
\label{section:rml_test_cases}
The RML test cases \citep{rml-test-cases} are a set of test cases to evaluate the conformity of an RML processor. Though these test cases are not a perfect match, they offer expected outputs for certain inputs and mapping rules, making them a good starting point for testing our implementation. The test cases are designed with edge cases in mind, and while they are an appropriate set of test cases to test a mapper, using them to test inversion is stretching their purpose a bit. As they are made to check if the full specification is implemented, many tests having duplicate inputs and outputs with differently written mapping rules to test the lexer, which we didn't make ourselves. They also do not conform to the limitations discussed in section \ref{section:limitations}. As such, if we were to simply take these test cases and invert them, we will get many generated source files that do not match the originals. Instead we try generating the knowledge graph again from our generated source files and compare the results with the original knowledge graph. We test the CSV and JSON test cases, as those are the ones we implemented.

\subsection{CSV test cases}
For the CSV test cases 23 out of 32 tests pass, the full breakdown can be found in figure \ref{itemize:rml_test_cases}. Most of the failures are limitations of the mapping processor, which only future update to the Morph-KGC library or a change to a different processor could solve. Another failure is due to data being stored in a blank node identifier, but no guarantees are made about the blank node's identifier in the RDF specification. For specific triple stores, it might be possible to retrieve the data from the identifier, but it is not generally possible. The other two failures are due to the data not being stored at the subject, but being used for a join condition. In this specific case the data could be retrieved as the join condition is an 'equals' function and the data is stored in the joined subject.

\begin{figure}[h]
    \centering
    \fbox{
        \begin{minipage}{\textwidth}
            Out of 32 tests:
            \begin{itemize}
                \item 23 tests pass
                \item 2 fail due to data being stored in a blank node
                \item 2 fail due to data not being directly stored at the subject, but being used for a join condition
                \item 5 fail because the mapping processor does not support them
            \end{itemize}
        \end{minipage}
    }
    \caption{Results of the CSV RML test cases}
    \label{itemize:rml_test_cases}
\end{figure}

\subsection{JSON test cases}
For the JSON test cases 24 out of 34 tests pass, the full breakdown can be found in figure \ref{itemize:rml_test_cases}. The failures are similar to the CSV test cases, aside from an extra failure due to the mapping containing no references, crashing the templating engine (for CSV an empty file is generated without crashing)

\begin{figure}[h]
    \centering
    \fbox{
        \begin{minipage}{\textwidth}
            Out of 34 tests:
            \begin{itemize}
                \item 24 tests pass
                \item 2 fail due to data being stored in a blank node
                \item 1 fails because the mapping contains no references, crashing the templating engine
                \item 2 fail due to data not being directly stored at the subject, but being used for a join condition
                \item 5 fail because the mapping processor does not support them
            \end{itemize}
        \end{minipage}
    }
    \caption{Results of the JSON RML test cases}
    \label{itemize:rml_test_cases}
\end{figure}


\section{LUBM4OBDA}
\label{section:lubm4obda}
The \acrfull{lubm4obda} benchmark \citep{LUBM4OBDA} is an extension of the \acrfull{lubm} benchmark \citep{LUBM}. Instead of generating OWL data it generates sql data, which paired with R2RML and RML mappings can be used to test \acrshort{odba} systems. We use the generated sql data to test the performance of our implementation on different scales. As we have not implemented a database module to recreate databases from a graph we instead reconstruct the views over the database which are used in the mappings. Comparing the speed the inversion is done for different scales of the benchmark gives us an idea of how well our implementation scales.

\subsection{Correctness}
We compare the generated source files with the views of the mapping. We find that 15 out 22 source files are successfully generated. The issues are caused by the mapping rules having duplicate subject-predicate-object maps generated by different sources without constants to differentiate them. An example conflict can be seen in Listing \ref{listing:lubm4obda_conflict}. As this is a flaw in the mapping rules, we can not mitigate this issue.

\begin{lstlisting}[caption={Example of a duplicate mapping pattern in the LUBM4OBDA benchmark}, captionpos=b, label={listing:lubm4obda_conflict}, basicstyle=\small, frame=single]
<#GraduateStudentAdvisor>
    rml:logicalSource [ 
        rml:source "graduatestudentadvisor.csv" ;
        rml:referenceFormulation ql:CSV ;
    ];
    rr:subjectMap [
        rr:template "http://www.department{dnr}.university{unr}.edu/{gname}";
    ];
    rr:predicateObjectMap [
        rr:predicate ub:advisor;
        rr:objectMap [ rr:template 
                "http://www.department{dnr}.university{unr}.edu/{fname}" ];
    ].

<#UndergraduateStudentAdvisor>
    rml:logicalSource [ 
        rml:source "undergraduatestudentadvisor.csv" ;
        rml:referenceFormulation ql:CSV ;
    ];
    rr:subjectMap [
        rr:template "http://www.department{dnr}.university{unr}.edu/{ugname}";
    ];
    rr:predicateObjectMap [
        rr:predicate ub:advisor;
        rr:objectMap [ rr:template 
                "http://www.department{dnr}.university{unr}.edu/{fname}" ];
    ].
\end{lstlisting}

\subsection{Performance}
We run our program on the LUBM4OBDA benchmark for different scales. The test is run on a machine with a Ryzen 7 7800x3D processor and 64GB of RAM. We use the free version of GraphDB as our triple store, this limits us to single threaded performance. The results can be found in Table \ref{table:lubm4obda_performance}. We find that the time it takes to invert the mappings scales linearly with the scale of the benchmark. The time spent within the program for data retrieval is minimal by design, leaving the majority of the computations to the triple store. As such, inverting the mappings is mostly dependent on the triple store's performance. The time required to convert the mappings to the query is minimal, even at the smallest scale it constitutes less than 0.2\% of the total time. For CSV files, minimal conversion is needed to transform the data to the source files. Although this time scales linearly with the benchmark scale, it accounts for less than 0.1\% of the total time.

\begin{table}[h]
    \centering
    \begin{tabular}{|c|c|}
        \hline
        \textbf{Scale} & \textbf{Time} \\
        \hline
        1              & 12.49s        \\
        10             & 127.61s       \\
        100            & 1379.26s      \\
        1000           & 13826.61s     \\
        \hline
    \end{tabular}
    \caption{\centering Performance on the LUBM4OBDA benchmark}
    \label{table:lubm4obda_performance}
\end{table}

\section{GTFS-Madrid-Bench}
\label{section:gtfs-madrid-bench}
The GTFS-Madrid-Bench \citep{gtfs-bench} benchmark is a benchmark for benchmark evaluating declarative KG construction engines. The data sources are based on the \acrfull{gtfs} data files of the subway network of Madrid. This data can be transformed into several formats such as CSV, JSON, SQL and XML. A scaling factor can also be applied to the data, allowing for different sizes of the benchmark. We use the CSV and JSON data sources to test our implementation on various scales. This is a good benchmark to test the performance of our JSON templating engine, as we have a baseline for the duration of the data retrieval with the CSV files.

\subsection{Correctness}
Comparing the generated to the original source files, we find that 7 out of 10 source files fully match for CSV, and 5 out of 10 source files match for JSON. The extra mismatches in JSON are due to the formatting of numbers, which is not guaranteed to be the same. In the original data the number are formatted as integers even though they are converted to doubles in the RDF, when converting back to JSON the exponent notation is used instead.

Of the three remaining mismatches, two are caused by part of the data only being used for a join condition. The last one is a bug caused by the mapping doing a join between entities of the same type. This is something we failed to take into account when making the implementation.

\subsection{Performance}
The same testing setup as the LUBM4OBDA benchmark is used for the GTFS-Madrid-Bench. As such we observe the same limitation on the total performance due to the single threaded nature of the triple store. The results can be found in Table \ref{table:gtfs-madrid-bench_performance}. We find that the time to invert scales linearly with the scale of the benchmark.

% csv-1: 6.75s
% csv-10: 60.49s
% csv-100: 641.00s
% json-1: 12.08s
% json-10: 109.60s
% json-100: 1126.07s
\begin{table}[h]
    \centering
    \begin{tabular}{|c|c|c|}
        \hline
        \textbf{Scale} & \textbf{CSV} & \textbf{JSON} \\
        \hline
        1              & 6.75s        & 12.08s        \\
        10             & 60.49s       & 109.60s       \\
        100            & 641.00s      & 1126.07s      \\
        \hline
    \end{tabular}
    \caption{\centering Performance on the GTFS-Madrid-Bench benchmark}
    \label{table:gtfs-madrid-bench_performance}
\end{table}
%%%%%%%%%%%%%%%%%%%%%%%%%%%%%%%%%%%%%%%%%%%%%%%%%%%%%%%%%%%%%%%%%%% 
%                                                                 %
%                            CHAPTER                              %
%                                                                 %
%%%%%%%%%%%%%%%%%%%%%%%%%%%%%%%%%%%%%%%%%%%%%%%%%%%%%%%%%%%%%%%%%%% 
\chapter{Roadmap}
\label{chapter:conclusion}
This thesis is far from done, as such no conclusion can be made for now. Here we discuss what will be done in the second semester.

\section{Implementation}
We will improve the implementation in various ways:
\begin{itemize}
    \item Support for more referenceFormulations:
    \begin{itemize}
        \item \acrshort{json}-path
        \item XPath
    \end{itemize}
    \item Joins
    \item More robustness
    \item More in-depth hybrid approach between SPARQL and local processing
    \item Inverting query-based sources further by inverting the query.
\end{itemize}

\section{Evaluation}
The number of benchmarks used will be expanded to include: LUBM4OBDA, GTFS-Madrid-Bench, SDM-Genomic-dataset, and any other benchmark that has RML (or R2RML) mappings available. We will also look into creating a version of the RML test cases for inversion, as the ones designed to test the conformance of RML processors are not ideal for testing the inversion algorithm.


% Eventueel enkele appendices
%%%%%%%%%%%%%%%%%%%%%%%%%%%%%%
%\appendix
\input{bijlage_1}

% Bibliografie: referenties. De items zitten in bibliografie.bib
%%%%%%%%%%%%%%%%%%%%%%%%%%%%%%%%%%%%%%%%%%%%%%%%%%%%%%%%%%%%%%%%%
% Indien je ook de niet geciteerde werken in je bibliografie wil opnemen, commentarieer dan onderstaande regel uit!
\nocite{*}
\bibliographystyle{apalike}
\bibliography{bibliografie}

\end{document}

% Back cover: change according to the correct campus
%\includepdf{private/back_fiiw_denayer.pdf}
\includepdf{private/back_fiiw_denayer_eng.pdf} % For the English version
%\includepdf{private/back_fiiw_geel.pdf}
% \includepdf{private/back_fiiw_geel_eng.pdf} % For the English version
%\includepdf{private/back_fiiw_gent.pdf}
% \includepdf{private/back_fiiw_ghent_eng.pdf} % For the English version
%\includepdf{private/back_fiiw_brugge.pdf}
% \includepdf{private/back_fiiw_bruges_eng.pdf} % For the English version
%\includepdf{private/back_fiiw_groept.pdf}
% \includepdf{private/back_fiiw_groupt_eng.pdf} % For the English version

% \end{document}