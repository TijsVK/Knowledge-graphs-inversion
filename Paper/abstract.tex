Knowledge graphs are gaining traction nowadays and more and more companies use them, such as Amazon, Bosch, IKEA, Facebook, Google, LinkedIn, SIEMENS, Zalando, etc. Most knowledge graphs are nowadays constructed from other heterogeneous data sources, such as tables in relational databases, data in XML files, or JSON format derived from a Web API. While the construction of knowledge graphs from heterogeneous data has been thoroughly investigated so far, the inverse, namely constructing raw data from knowledge graphs hasn't been explored in depth yet. 

In this thesis, we propose a method to invert knowledge graphs back to raw data. We will use the same rules used to construct the knowledge graph to do the inverse. Our method is split into two parts. First, a template is created by inverting the value retrieval of the mappings. For each supported source file this requires a tailored implementation. Secondly, the data is retrieved from the knowledge graph by using the mappings to generate a query for each iterator in the mappings. In the end, both these parts are combined, putting the data into the template.

Access to the source can be given upon request at \href{mailto:tijs.vankampen@student.kuleuven.be}{tijs.vankampen@student.kuleuven.be}. 

For this thesis, Github Copilot was used for assistance with the coding of the implementation, and rewording of sentences in the paper for readability.

