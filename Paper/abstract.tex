Knowledge graphs are gaining traction nowadays and more and more companies use them, such as Amazon, Bosch, IKEA, Facebook, Google, LinkedIn, SIEMENS, Zalando, etc. Most knowledge graphs are nowadays constructed from other heterogeneous data sources, such as tables in relational databases, data in XML files, or JSON format derived from a Web API. While the construction of knowledge graphs from heterogeneous data has been thoroughly investigated so far, the inverse, namely constructing raw data from knowledge graphs hasn't been explored in depth yet. 

In this thesis, we propose a method to invert knowledge graphs back to raw data. We will use the same rules used to construct the knowledge graph to do the inverse. Our method is split into two parts. First, a template is created by inverting the value retrieval of the mappings. For each supported source file this requires a tailored implementation. Secondly, the data is retrieved from the knowledge graph by using the mappings to generate a query for each iterator in the mappings. In the end, both these parts are combined, putting the data into the template.

Access to the source can be given upon request at \href{mailto:tijs.vankampen@student.kuleuven.be}{tijs.vankampen@student.kuleuven.be}. 

For this thesis, generative AI is used for assistance with the coding of the implementation, and rewording of sentences in the paper for readability.

% In this thesis, we will explore the field of few-shot object detection to find out if recent advancements in the field have made it performant enough to be able to detect and count shells on a beach.

% Few-shot object detection is a newer field of research in computer vision that has been gaining traction in the last few years. 
% As opposed to the related field of image classification, where the goal is to classify an image into a class, object detection aims to detect and classify objects in an image. This is a more complicated task, as the network has to not only classify the objects but also localize the objects in the image. In addition the limitation of not having a lot of images of the objects to work with results in not having sufficient information on the objects to detect them. The only way to achieve reliable detection is to make up for the gap in our knowledge in a different way. This is where the field of few-shot object detection comes in. It allows us to first gain generic knowledge by learning from a large dataset and then fine-tuning that knowledge to our specific task by learning from a small dataset.



% The implementation and source for this paper can be found at https://github.com/TijsVK/Knowledge-graphs-inversion.

% For this thesis, generative AI will be used for assistance with the implementation, and rewording of sentences and sections for readability.

% \textbf{Keywords}: Few-shot object detection, object detection, neural networks, shells, beach, computer vision, machine learning