%%%%%%%%%%%%%%%%%%%%%%%%%%%%%%%%%%%%%%%%%%%%%%%%%%%%%%%%%%%%%%%%%%% 
%                                                                 %
%                            CHAPTER                              %
%                                                                 %
%%%%%%%%%%%%%%%%%%%%%%%%%%%%%%%%%%%%%%%%%%%%%%%%%%%%%%%%%%%%%%%%%%% 
\chapter{Roadmap}
\label{chapter:conclusion}
This thesis is far from done, as such no conclusion can be made for now. Here we discuss what will be done in the second semester.

\section{Implementation}
We will improve the implementation in various ways:
\begin{itemize}
    \item Support for more referenceFormulations:
    \begin{itemize}
        \item \acrshort{json}-path
        \item XPath
    \end{itemize}
    \item Joins
    \item More robustness
    \item More in-depth hybrid approach between SPARQL and local processing
    \item Inverting query-based sources further by inverting the query.
\end{itemize}

\section{Evaluation}
The number of benchmarks used will be expanded to include: LUBM4OBDA, GTFS-Madrid-Bench, SDM-Genomic-dataset, and any other benchmark that has RML (or R2RML) mappings available. We will also look into creating a version of the RML test cases for inversion, as the ones designed to test the conformance of RML processors are not ideal for testing the inversion algorithm.