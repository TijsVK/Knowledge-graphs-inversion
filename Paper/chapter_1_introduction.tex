%%%%%%%%%%%%%%%%%%%%%%%%%%%%%%%%%%%%%%%%%%%%%%%%%%%%%%%%%%%%%%%%%%% 
%                                                                 %
%                            CHAPTER                              %
%                                                                 %
%%%%%%%%%%%%%%%%%%%%%%%%%%%%%%%%%%%%%%%%%%%%%%%%%%%%%%%%%%%%%%%%%%% 

\chapter{Introduction}
\label{chapter:introduction}

% TODO: Add introduction text

The earliest academic definition of a knowledge graph can be found in a 1974 article as \begin{quote}
    A mathematical structure with vertices as knowledge units connected by edges that represent the prerequisite relation \citep{Marchi1974,bergman2019common}
\end{quote} 

The idea of expressing knowledge in a graph structure predates even this definition, with the concept of semantic networks \citep{Richens1956PreprogrammingFM}. % this is in the ago of punch card computers, so quite impressive 
However, the term knowledge graph only became well-known after Google announced they were using a knowledge graph to enhance their search engine in 2012 \citep{singhal2012introducing}. 
Knowledge graphs are used to make search engines, chatbots, question answering systems, etc more intelligent by injecting knowledge into them \citep{SurveyOnKGs}. 
These knowledge graphs are constructed by extracting information from various sources, both unstructured sources such as text (using natural language processing) and (semi-)structured sources such as databases, CSV, XML, JSON, RDF (using mapping languages). Many mapping languages exist, some with a specific purpose, such as R2RML \citep{Das:12:RRR} for relational databases, XSPARQL \citep{Bischof2012} for XML. Others are more general, such as RML \citep{dimou_ldow_2014} and D2RML \citep{Chortaras2018D2RMLIH}, having the ability to map from multiple sources in different formats. 
To achieve this these mapping languages use a declarative approach, where the user specifies the mapping rules, and the implementation of the mapping language takes care of the actual mapping. Creating these mapping rules is often done by hand. Tools do however exist to help with this, like RMLEditor \citep{heyvaert_jws_2018} and YARRRML \citep{10.1007/978-3-319-98192-5_40}. Alternatively some tools exist for automatically generating these mapping rules \textit{cite some tools}.

Most existing programs and services aren't build to consume knowledge graphs, so to use the data the knowledge graph needs to be converted to a different format compatible with the system. This can currently be done with SPARQL \citep{Seaborne:08:SQL} for tabular data, or XSPARQL \citep{Bischof2012} for XML. The current state of the art for this is quite limiting though, with methods limited to a single file type.

This paper aims to improve this situation by extending a general mapping language with the ability to invert the mapping rules, i.e. mapping the RDF knowledge graph back to the original data format. We choose to extend RML \citep{dimou_ldow_2014} as it fulfills that requirement, and its end-to-end characteristics make it a good candidate for this task. Practically we will extend the Morph-KGC \citep{arenas2022morph} implementation.


\section{Thesis outline}
This thesis aims to explore the possibility of inverting knowledge graphs back to their original data format using RML mapping rules. To achieve this we will first look at the current state of the art in chapter \ref{chapter:related_work}. We will take a closer look at the technologies used like RDF, SPARQL, and RML. We will also look at the current state of the art for inverting knowledge graphs. In chapter \ref{chapter:implementation} we will look at our implementation of the inversion algorithm. We will look at the algorithm itself, and the implementation details. In chapter \ref{chapter:evaluation} we will evaluate our implementation using a number of test cases as well as some real world use cases. Finally in chapter \ref{chapter:conclusion} we will conclude this thesis, and look at possible future work.